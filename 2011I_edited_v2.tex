%%
%% This is file `elsarticle-template-harv.tex',
%% generated with the docstrip utility.
%%
%% The original source files were:
%%
%% elsarticle.dtx  (with options: `harvtemplate')
%%
%% Copyright 2007, 2008 Elsevier Ltd.
%%
%% This file is part of the 'Elsarticle Bundle'.
%% -------------------------------------------
%%
%% It may be distributed under the conditions of the LaTeX Project Public
%% License, either version 1.2 of this license or (at your option) any
%% later version.  The latest version of this license is in
%%    http://www.latex-project.org/lppl.txt
%% and version 1.2 or later is part of all distributions of LaTeX
%% version 1999/12/01 or later.
%%
%% The list of all files belonging to the 'Elsarticle Bundle' is
%% given in the file `manifest.txt'.
%%
%% Template article for Elsevier's document class `elsarticle'
%% with harvard style bibliographic references
%% SP 2008/03/01

\documentclass[preprint,12pt]{elsarticle}

%% Use the option review to obtain double line spacing
%\documentclass[authoryear,preprint,review,12pt]{elsarticle}

%% Use the options 1p,twocolumn; 3p; 3p,twocolumn; 5p; or 5p,twocolumn
%% for a journal layout:
%% \documentclass[final,1p,times]{elsarticle}
%% \documentclass[final,1p,times,twocolumn]{elsarticle}
%% \documentclass[final,3p,times]{elsarticle}
%% \documentclass[final,3p,times,twocolumn]{elsarticle}
%% \documentclass[final,5p,times]{elsarticle}
%% \documentclass[final,5p,times,twocolumn]{elsarticle}

%How Do I Get LaTeX to Double-Space?
\renewcommand{\baselinestretch}{1}
% For other spacing (ie triple) just change the 1 to 2.

%% if you use PostScript figures in your article
%% use the graphics package for simple commands
%% \usepackage{graphics}
%% or use the graphicx package for more complicated commands
\usepackage{graphicx}
%% or use the epsfig package if you prefer to use the old commands
%% \usepackage{epsfig}

%% The amssymb package provides various useful mathematical symbols
\usepackage{amssymb}
%% The amsthm package provides extended theorem environments
\usepackage{amsthm}

%\usepackage{algorithm}
%\usepackage{algorithmic}

%% The lineno packages adds line numbers. Start line numbering with
%% \begin{linenumbers}, end it with \end{linenumbers}. Or switch it on
%% for the whole article with \linenumbers.
\usepackage{lineno}
\usepackage{amsmath}
\usepackage{multirow}

%\usepackage[fleqn]{amsmath}
%\usepackage{array}
%\usepackage{units}
%\usepackage{tikz}
%\usepackage{graphicx}
\usepackage{booktabs}
%\usepackage{setspace}
%\usepackage{pstricks,framed}
\usepackage{hyperref}  % \url


\journal{ Applied Soft Computing, Volume 12, Issue 8, August 2012, Pages 2631�C2637 }

\begin{document}

\linenumbers

\begin{frontmatter}
%% Title, authors and addresses
%% use the tnoteref command within \title for footnotes;
%% use the tnotetext command for theassociated footnote;
%% use the fnref command within \author or \address for footnotes;
%% use the fntext command for theassociated footnote;
%% use the corref command within \author for corresponding author footnotes;
%% use the cortext command for theassociated footnote;
%% use the ead command for the email address,
%% and the form \ead[url] for the home page:
%% \title{Title\tnoteref{label1}}
%% \tnotetext[label1]{}
%\author{Yi Chen\corref{cor1}\fnref{label1=e1}}
%\ead{ yichen@mech.gla.ac.uk }
%\author{Matthew P. Cartmell\corref{cor1}\fnref{label1=e1}}
%\ead{ matthewc@mech.gla.ac.uk}
%%\ead[url]{home page}
%% \fntext[label2]{}
%% \cortext[cor1]{}
%% \address{Address\fnref{label3}}
%% \fntext[label3]{}

%\title{ Hybrid Fuzzy Skyhook Surface Sliding Mode Control
%on Semi-active Vehicle Suspension Systems for Ride Comfort
%Enhancement }

\title{Exchange Rates Determination Based on Genetic Algorithms using Mendel's Principles:
Investigation and Estimation under Uncertainty}

% not more than 10 words
%\author{Yi Chen \corref{cor1}\fnref{label1=e1}}
\author[First,Second]{Yi Chen \corref{cor1}\fnref{fn1}}
%\author[First]{Yi Chen \corref{cor1}\fnref{fn1}}
\author[Third,Fourth]{Guangfeng Zhang}
%\author[First]{Yi Chen\corref{1}\fnref{First}}
%\author[Second]{Second B. Author, Jr.}
%\author[Third]{Third C. Author}
%\address{Department of Mechanical Engineering, University of Glasgow, Glasgow, G12 8QQ, UK (e-mail: yichen@mech.gla.ac.uk)}
%\fntext[label1=e1]{fax: 44(0)-141-330-4343 }
\cortext[cor1]{e-mail: leo.chen.yi@live.co.uk,
%        tel: +44(0)-141-330-2477,
%        fax: +44(0)-141-330-4343. 
}

\address[First]{School of Mechatronics Engineering,
 University of Electronic Science and Technology of China, Chengdu,
 611731, China}
 
\address[Second]{Department of Mechanical Engineering, University of Glasgow, Glasgow, G12 8QQ, UK
%, (e-mail: yichen@mech.gla.ac.uk)
}

\address[Third]{Department of Economics and Finance, School of Social Sciences, Brunel University, Uxbridge, England, UB8 3PH, UK
%(e-mail: guangfeng.zhangATbrunel.ac.uk)
%\address[Third]{Department of Economics, University of Glasgow, Glasgow, G12 8RT, UK
%, (e-mail: guangfeng.zhang@brunel.ac.uk, g.zhang.1@research.gla.ac.uk)
}

\address[Fourth]{Department of Economics, University of Glasgow, Glasgow, G12 8RT, UK
%, (e-mail: guangfeng.zhang@brunel.ac.uk, g.zhang.1@research.gla.ac.uk)
}

%\address[Third]{Electrical Engineering Department,
%   Seoul National University, Seoul, Korea, (e-mail: author@snu.ac.kr)}

%% use optional labels to link authors explicitly to addresses:
%\author{Yi Chen\corref{cor1}\fnref{label1=e1}}
%\author{Yi Chen\corref{cor1}}
%\address{Department of Mechanical Engineering, University of Glasgow, Glasgow, G12 8QQ, UK}
%\ead{ yichen@mech.gla.ac.uk, fax: 44(0)-141-330-4343 }

\fntext[fn1]{These authors contributed equally to this work.}

\begin{abstract}
\label{abstract}
%% Text of abstract
A genetic algorithm using Mendel's principle (Mendel-GA), in which the random
assignment of alleles from parents to offsprings is implied by the
Mendel genetic operator, is proposed for the exchange rates determination problem.
Besides the traditional genetic operators of selection, crossover, and mutation, Mendel's principles are
included, in the form of an operator in the genetic algorithm's evolution process. In
the quantitative analysis of exchange rates determination, the
Mendel-GA examines the exchange rate fluctuations at the
short-run horizon. Specifically, the aim is to revisit the
determination of high-frequency exchange rates and examine the
differences between the method of genetic algorithms and that of the
traditional estimation methods. A simulation with a given initial
conditions has been devised in $MATLAB$, and it is shown that the
Mendel-GA can work valuably as a tool for the exchange rates
estimation modelling with high-frequency data.
\end{abstract}

\begin{keyword}
%% keywords here, in the form: keyword \sep keyword
genetic algorithm \sep Mendel's principle  \sep uncertainty \sep
estimation \sep high-frequency \sep exchange rates \sep regression
%% PACS codes here, in the form: \PACS code \sep code
%% MSC codes here, in the form: \MSC code \sep code
%% or \MSC[2008] code \sep code (2000 is the default)
\end{keyword}

\end{frontmatter}

%\linenumbers

\section{Introduction}
\label{Introduction}

The modelling of exchange rates movements is a challenging task in
international finance. A strong consensus in academic research is
that macroeconomic fundamentals have no explanatory power for
exchange rates fluctuations in the short run \cite{Obstfeld2000,Rogoff1996}. In contrast, microstructure approaches focus on
how is information concerning the macro fundamentals,
non-fundamentals and its transfers in the foreign exchange market,
and impacting the movement of exchange rates. Empirical evidence
demonstrates the significant positive link between exchange rates and
their corresponding contemporaneous order flow, which is defined as
the net value between buyer-initiated trade and seller-initiated
trade \cite{Lyons2001,Killeen2001,Payne2003}.

Other evolutionary computation (EC) methods were proposed for
exchange rates analysis, and other financial studies. In 1996, Hann and
Steurer \cite{Hann1996} analysed the influences of data frequency on
American Dollars/Deutsch Mark forecasting by artificial neural
networks (ANN), in which the studies reported that the ANN do not greatly
improve the forecasting accuracy when monthly data is applied.
In 2003, Qi and Wu \cite{Qi2003} proposed a multi-layer feed forward
network to forecast exchange rates, the numerical results of which
concluded that the ANN can not perform efficiently in out-of-sample
forecast accuracy. In 2007, Yadav, Kalra, and John \cite{Yadav2007}
applied standard multilayer neural network (SMN) to predict a set of
time-series data for an exchange rate prediction from 2002 to 2004.

In 2005, Rimcharoen, Sutivong, and Chongstitvatana
\cite{Rimcharoen2005} proposed a method of adaptive evolution
strategies (ES) for the prediction of the stock exchange of
Thailand, in which the GA method was combined with the ES method.
No further reports about ES for prediction of exchange rates studies
have been made.

A differential evolution (DE) algorithm, combining the strengths of
multiple strategies, was proposed by Worasucheep and
Chongstitvatana \cite{Worasucheep2009} in 2009, but there is no
further studies on the exchange rates determination by the DE or DE
related methods.

The particle swarm optimisation (PSO) method is one of the swarm
intelligence algorithms, which is a population-based search
algorithm following the social behaviour of individuals
(particles) moving among a multi-dimensional searching space. The
PSO method was applied to stock markets forecasting, by working with ANN,
by Nenortaite and Simutis \cite{Nenortaite2004} in 2004, and by Zhao
and Yang \cite{Zhao2009} in 2009, but neither reports on the PSO
applications for exchange rates prediction.

Genetic Algorithms (GA) were introduced in the 1970s by John Holland
\cite{Holland1975} at the University of Michigan. Inspired by
Darwin's theory of evolution, they apply three basic genetic
operators - selection, crossover, and mutation - to a population
of individuals. The practical problems are often characterised by
several non-commensurable and competing measures of performance or
objectives, with a number of restrictions imposed on the decision
variables. The choice of a suitable compromise solution from all
non-inferior alternatives is not only problem-dependent, it
generally depends also on the subjective preferences of a decision
agent. Thus, the final solution to the problem is the result of both
an optimisation process and a decision process\cite{Chen_JAP_2011,Chen_AES_2011,Chen_OE_2011,Chen_CEC_2007}.

%http://encarta.msn.com/dictionary_/Daffodil.html

%\begin{figure}[h!]
%\centering
%\includegraphics[scale=0.15]{Mendel_Daffodil}
%\caption{Daffodils with different hybrid characters}
%\label{Mendel_Daffodil}
%\end{figure}

%Since ancient time, farmers and herders have been selectively
%breeding their plants and animals to produce more useful hybrids.
%Knowledge of the genetic mechanisms finally came as a result of
%careful laboratory breeding experiments carried out and published by
%Mendel in 1860s \cite{Mendel1865} \cite{O'Neil2009}. Through the
%selective cross-breeding of common pea plants over many generations,
%Mendel discovered that certain traits show up in offspring without
%any blending of parental characteristics. For example, the pea flowers
%are either purple or white - intermediate colors do not appear in the
%offspring of cross-pollinated pea plants. Although Mendel's research
%was with plants, the basic underlying principles of heredity that he
%discovered apply also to people and animals, because the
%mechanisms of heredity are essentially the same for all complex life
%forms. For example, Figure \ref{Mendel_Daffodil} shows Daffodils with
%different hybrid characters, such as colour - white (or whiteish) daffodils,
%green, yellow, pink, orange, etc. - shape, and size.

In recent years, a lot of literature has been proposed in the area
of GA using Mendel's principles \cite{Mendel1865,O'Neil2009,Song1999,Park2001,Kadrovach2001,Kadrovach2002}. In this paper, a new GA
method using Mendel's principles (Mendel-GA) is proposed,
which includes the following differences to the standard GA method and
previous research:
\begin{list}{$$}
%
\item (1) In
this paper, the Mendel operator is inserted after the selection
operator, which can thus take advantage of the Mendel operator's local search
ability, as shown in Figure \ref{Mendel_GA_workflow}. In previous researches \cite{Song1999,Park2001,Kadrovach2001,Kadrovach2002}, the Mendel operator was
inserted into the GA process after the mutation operator. Based on mutation probability $P_m$, mutation may generate an unstable
population, which will lead to polluted outputs for the whole evolutionary processes biologically and mathematically\cite{Ishibuchi2002,Furi2005}. Typically, the mutation operator is a randomly introduced changing of a binary bit from a `0' to
a `1', and vice versa. The basic method of mutation is able to generate new recombination of improved solutions at a given rate, but the possibility of damage to the dominated population, loss of good solutions and convergence trend also occurs\cite{Zitzler1999,NAS2004,Haupt2004}. The Mendel operator will amplify such an unstable population
with its local search ability from a micro evolutionary point of view. 
%
\item (2) Mendel's principles are represented by the Mendel
operator, which is easily synchronised with the multi-objective
GA processes, such as multiple objective genetic algorithm (MOGA)
\cite{MOGA1993}, niched pareto genetic algorithm (NPGA)
\cite{NPGA1994}, nondominated sorting genetic algorithm (NSGA)
\cite{NSGA1994} and nondominated sorting genetic algorithm II
(NSGAII) \cite{NSGAII2002}.
%
\item (3) The standard GA is based on Darwin's theory, which is
represented by the differential survival, and reproductive success;
in the Mendel-GA, Mendel's law is indicated by the equal
gametes, which unite at random to form equal zygotes and reproduce
equal plants throughout all stages of the life cycle.
%
\end{list}


Exchange rates determination has been regarded as one of the most
challenging applications of high frequency time series trading
\cite{Lyons2001,Killeen2001,Payne2003,Tenti1996,Gujarati2004}, and, to provide the investors and researchers with more
precise predictions, some different models have
been depicted in which the prices follow a random walk
phenomenon. This is suitable for GA with stochastic and non-linear
searching ability.

The Mendel-GA will be applied to the studies of empirical
analysis on exchange rates determination, which can provide an
evolutionary and computational method to the exchange rates
determination problem. Specifically, it attempts to compare the
performance of the Mendel-GA and the traditional estimation methods, for
instance, the ordinary least square(OLS) or the linear least squares
(LS) estimation. The OLS and LS are methods for estimating the
unknown parameters in a linear regression model. These methods
minimise the sum of squared distances between the observed
responses in the data-set, and the responses predicted by the linear
approximation. Compared with the OLS or LS, the Mendel-GA, by the evolutionary process,
can handle linear and non-linear models with higher complexity, and it is flexible to be an active
optimisation solver for switching from one prediction model to the
others.

\section{Exchange Rates Determination Model}
\label{Exchange rates determination}


In 2001, Kileen, Lyon, and Moore \cite{Killeen2001} found the
co-integration relationship between exchange rates and cumulative
order flow (COF), which is the proximate determinant of price in all
microstructure models, and in 2003 Payne \cite{Payne2003} used
non-standard vector autocorrelation to examine the causality of
exchange rates from the order flow.

The order flow is defined as the net of buyer and seller initiated
currency transactions, which is taken as a measure of net
buying pressure \cite{Lyons2001}. According to previous reports
\cite{Obstfeld2000,Killeen2001,Payne2003,Reitz2007,Rime2008}, the order flow is intimately
related to a broad set of current and expected macroeconomic
fundamentals, and as such the order flow is regarded as a useful predictor
of the movements in exchange rates.

Microstructure approaches use order flow to proxy the information
reflecting the movement of exchange rates. An intensive study by Lyons
\cite{Obstfeld2000} in 2000 demonstrates that order flow contains
information concerning the movements of exchange rates.


By starting from the conventional exchange rates theories, the
exchange rates can be expressed as the discounted present value of
current and expected fundamentals as given in the equation
(\ref{Mendel_exchange_rates_st}) \cite{Obstfeld2000,Killeen2001,Payne2003,Rime2008,Andersen2003}.

\begin{equation}
s_t = (1-b)\sum^\infty_{q=0} b^q E^m_t f(x_{t+q})
\label{Mendel_exchange_rates_st}
\end{equation}

Where,
\begin{list}{$$}
%
\item $\circ$ $s_t$ is the spot exchange rate,
which is defined as the domestic price of foreign currency;
%
\item $\circ$ $b$ is the discount factor;
%
\item $\circ$ $q$ is the future periodic factor;
%
\item $\circ$ $f(x_{t})$ denotes the fundamentals at time $t$;
%
\item $\circ$ $x_t$ is the contemporaneous order flow;
%
\item $\circ$ $E^m_t f(x_{t+q})$  is the market-makers's expectations about
future fundamentals at $q$-periods after time $t$, conditional on information
available at time $t$;
\end{list}

Specifically, the relation between the spot exchange rates $s_t$
and the contemporaneous order flow $x_t$ can be specified as follows,
by iterating equation (\ref{Mendel_exchange_rates_st}) forward and
rearranging terms one as given in equation
(\ref{Mendel_exchange_rates_deltast}): % HERE

\begin{equation}
\Delta s_{t+1} = \displaystyle \frac{1-b}{b}\left( s_t - E^m_t
f(x_t) \right) + {\varepsilon_{t+1}}
\label{Mendel_exchange_rates_deltast}
\end{equation}

where,
%
\begin{list}{$$}
%
\item $\circ$ $\Delta$ denotes the first-difference of the series;
%
\item $\circ$ $\varepsilon_t$ is the disturbance term, as stated in
equation (\ref{Mendel_exchange_rates_varepsilon})
\cite{Bacchetta2006}, which shows that the future exchange rates
change is a function of the gap between the current exchange rates
and the expected current fundamentals. It is also a term
that captures changes in expectations about fundamentals;

\begin{equation}
\varepsilon_{t+1} = (1-b)\sum^\infty_{q=0} b^q \left( E^m_{t+1}
f(x_{t+q+1}) - E^m_{t} f(x_{t+q+1}) \right)
\label{Mendel_exchange_rates_varepsilon}
\end{equation}
%
\end{list}



Purchasing foreign currency increases the demand of the foreign
currency, leading to an appreciation of the foreign currency, and a
depreciation of the domestic currency, i.e., the spot exchange rates
$s_t$ get increased. We expect a positive correlation between the
spot exchange rates and the corresponding order flow.



\section{ Genetic Algorithm using Mendel's Principles  }
\label{Multi-objective Genetic Algorithm using Mendel's Principles}

Mendel's principles (some biologists refer to Mendel's `principles'
as `laws') have always applyed to the genetically reproducing
creature in a natural environment. Mendel's conclusions from these pea
experiments can be summarized in two principles \cite{Mendel1865,O'Neil2009,Marks2008}:(i) the principle of segregation - for any particular trait, the
pair of alleles of each parent separate, and only one allele passes
from each parent on to an offspring;(ii) the principle of independent assortment - different pairs of
alleles are passed to offspring independently of each other.
%
%\end{list}

\begin{figure}[h!]
\centering
\includegraphics[scale=0.17]{Mendel_GA_workflow}
\caption{Genetic algorithm using Mendel's principles workflow}
\label{Mendel_GA_workflow}
\end{figure}

According to Mendel's experiments in plant hybridisation, the
striking regularity with which the same hybrid forms (colour, shape, etc.)
always reappeared whenever fertilisation took place between the same
species demanded further experiments to be undertaken. As shown in Figure \ref{Mendel_GA_workflow}, we have included Mendel's
principles in GA's overall evolution framework, which
can describe the accurate heredity meiosis process from a micro
evolution point of view in a GA process. A GA using Mendel's principles
can be defined as when a Mendel genetic operator is introduced into the GA
methods mentioned above, and then applied to the optimisation
studies and parameters estimation for the regression modelling of
the exchange rates determination.


%D:\Working\SGALAB1003\SGALAB_PPT\FIGS

\begin{figure}[h!]
\centering
\includegraphics[scale=0.15]{Mendel_parents_offspring}
\caption{Mendel operator: parents to offsprings }
\label{Mendel_parents_offspring}
\end{figure}

\begin{table}[h!]
%\begin{sidewaystable}[ht!]
  \centering
  \caption{Mendel punnet-square for 1-bit chromosome}
%   \scriptsize % re-set the font size for the table
    \begin{tabular}{r|c|ccc}
    \addlinespace
    \toprule
    \multicolumn{2}{c}{} & \multicolumn{3}{c}{$attrP_1$}  \\
%    \midrule
     \cmidrule{3-5}
    \multicolumn{2}{c} {{$attrO$}} &{{$D$}} & { {$H$}} & { { $R$}}\\ \cmidrule{3-5}
    { }     & { { $D$}} & $100\% D$ &   $\begin{matrix} 50\% D \\ 50\% H \end{matrix}$     & $100\% H$    \\
    { }     &  \\
    {$attrP_2$} & { { $H$}} & $\begin{matrix} 50\% D \\ 50\% H \end{matrix}$    & $\begin{matrix} 25\% D \\ 50\% H \\ 25\% R  \end{matrix}$    & $\begin{matrix} 50\% R \\ 50\% H \end{matrix}$  \\
    { }     &  \\
    { }     & { { $R$}} & $100\% H$    & $\begin{matrix} 50\% R \\ 50\% H \end{matrix}$    & $100\% R$   \\
       \bottomrule
    \end{tabular}
\label{Mendel_punnet_square}
\end{table}

\section{Mendel Operator }
\label{Mendel_Operator}

In this paper, the binary encoding method is utilised for the Mendel-GA's encoding/decoding operation. Typically, all bits(genes) of a chromosome could be encoded using the two binary alphabets `1' and `0', in which, `1' represents a gene is active and `0' is inactive.

As shown in Figure \ref{Mendel_parents_offspring}, `0' and `1' are
the two bit-values of the binary encoded chromosome, and the
chromosome length, or bits number, for the Mendel's genetic operator
is called Mendel's percentage ($MP$), where $MP$ is a
proportionality factor, included to balance the bits length of the
parental chromosome. %Clearly, $MP$ = 0 means no bit is involved
%in current generation of evolution, and $MP$ = 1 means all bits
%of the the parental chromosome are involved in the evolution, $MP$
%$\in$ [0,1] or [0,100$\%$].
The Mendel operator generates the offspring's chromosomes
($x_{{O}_i}$) from the parent's chromosomes ($x_{{P}_i}$) according
to every bit's attribute, which is defined by the
Punnet-square\cite{Mendel1865,O'Neil2009}. Two parental
chromosomes $x_{{P}_1}$ and $x_{{P}_2}$ can generate one child
chromosome $x_{{O}_1}$, as shown in Figure
\ref{Mendel_parents_offspring}; to obtain the second child
chromosome $x_{{O}_2}$, the productive process need to be repeated.


As stated in Table \ref{Mendel_punnet_square}, in the Punnet-square
for the Mendel operator, each chromosome bit is assigned an
attribute, which indicates the type of the gene's corresponding
character. There are three types of attributes of a gene: $D$(Dominant, the pure and dominant gene),
$R$(Recessive, the pure and recessive gene), and $H$(Hybrid, the
hybrid gene). $attrP_1$ and $attrP_2$ are the attributes of a
bit of the parental chromosomes $x_{{P}_1}$ and $x_{{P}_2}$, $attrO$
is the attribute of a bit of a children chromosome $x_{{O}_1}$ or
$x_{{O}_2}$. 

%%%By Table \ref{Mendel_punnet_square}, it can be
%%%clarified that:
%%%%\begin{framed}
%%%\begin{list}{$$}
%%%% Rule - 1
%%%\item $\langle1\rangle$ IF   $attrP_1$  = $D$ AND
%%%\begin{list}{$$}
%%%% 1
%%%        \item   IF  $attrP_2$  = $D$,
%%%                THEN $attrO$   = $100\% D$;
%%%% 2
%%%        \item   IF  $attrP_2$  = $H$,
%%%                THEN $attrO$   = $50\% D$ or $50\% H$;
%%%% 3
%%%        \item   IF  $attrP_2$  = $R$,
%%%                THEN $attrO$   = $100\% H$;
%%%%
%%%\end{list}
%%%
%%%% Rule - 2
%%%\item $\langle2\rangle$ IF   $attrP_1$  = $H$ AND
%%%\begin{list}{$$}
%%%% 1
%%%        \item   IF  $attrP_2$  = $D$,
%%%                THEN $attrO$   = $50\% D$ or $50\% H$;
%%%% 2
%%%        \item   IF  $attrP_2$  = $H$,
%%%                THEN $attrO$   = $25\% D$ or $50\% H$ or $25\% R$;
%%%% 3
%%%        \item   IF  $attrP_2$  = $R$,
%%%                THEN $attrO$   = $50\% R$ or $50\% H$;
%%%%
%%%\end{list}
%%%
%%%% Rule - 3
%%%\item $\langle3\rangle$ IF   $attrP_1$  = $R$ AND
%%%\begin{list}{$$}
%%%% 1
%%%        \item   IF  $attrP_2$  = $D$,
%%%                THEN $attrO$   = $100\% H$;
%%%% 2
%%%        \item   IF  $attrP_2$  = $H$,
%%%                THEN $attrO$   = $50\% R$ or $50\% H$;
%%%% 3
%%%        \item   IF  $attrP_2$  = $R$,
%%%                THEN $attrO$   = $100\% R$;
%%%%
%%%\end{list}
%%%\end{list}
%%%%\end{framed}

%\begin{algorithmic}
%  \IF{some condition is true}
%    \STATE do some processing
%  \ELSIF{some other condition is true}
%    \STATE do some different processing
%  \ELSIF{some even more bizarre condition is met}
%    \STATE do something else
%  \ELSE
%    \STATE do the default actions
%  \ENDIF
%\end{algorithmic}


%According to Mendel's principles, the hybridisation is the
%process of combining different attributes of all bits of the
%chromosomes to create the offspring bits with Punnet-square driven
%attributes. Compared with a crossover operator or a mutation
%operator, the Mendel operator performs with a Punnet-square driven
%behaviour.


%http://mechanismsevo.blogspot.com/2007/10/mendels-laws.html

\section{Fitness Function Definition}
\label{Fitness Function Design}


\subsection{Measure of the goodness of fit - the coefficient of determination}
\label{R_sq}

Regression analysis is the investigation of the functional
relationship between two or more variables, in which the coefficient
of determination($R^2$) is a quantity in the regression models used
to measure the proportion of total variability in the response
accounted for, by the model in Figure \ref{R2definition}
\cite{Gujarati2004,Montgomery2003}, and the larger values of
$R^2$(near unity) are considered the better of the variability in
the data, as given in equation
(\ref{Mendel_coefficient_determination}).


\begin{figure}[h!]
\centering
\includegraphics[scale=0.23]{R2definition}
\caption{The coefficient of determination($R^2$) definition
\cite{Gujarati2004,Montgomery2003}} \label{R2definition}
\end{figure}

\begin{equation}
R^2 \left( y_i,\hat{y}_i \right) = 1 -  \frac{SS_{err}}{SS_{tot}} =
1 - \frac{\sum\limits_{i = 1}^k{ \left( y_i - \hat{y}_i
\right)^2}}{\sum\limits_{i = 1}^k{ \left( y_i - \bar{y}_i
\right)^2}}
\label{Mendel_coefficient_determination}
\end{equation}

%http://en.wikipedia.org/wiki/Coefficient_of_determination
%In this form R2 is given directly in terms of the explained
%variance: it compares the explained variance (variance of the
%model's predictions) with the total variance (of the data).

where,
\begin{list}{$$}
%
\item $\circ$ $R^2$ is the coefficient of determination for the two given data sequences $y_i$ and $\hat{y}_i$, as shown in Figure \ref{R2definition}, $y_i$ is the dashed data and $\hat{y}_i$ is the dotted data;
%
\item $\circ$ $y_i$ is the observed data,
which is the exchange trade data \cite{Evans2002};
%
\item $\circ$ $\hat{y}_i$ is the specimen data,
which obtained from the Mendel-GA over the evaluating process;
%
\item $\circ$ $\bar{y}_i$ is the mean value of the observed data $y_i$;
%
\item $\circ$ $k$ is the sample size of both two given data sequences $y_i$ and $\hat{y}_i$.
%
\item $\circ$ $SS_{tot}$ is the total corrected sum of squares,
as given in equation (\ref{Mendel_coefficient_determination_SS_tot})
\cite{Gujarati2004,Montgomery2003};
%
\begin{equation}
SS_{tot} = \sum\limits_{i = 1}^k{ \left( y_i - \bar{y}_i \right)^2}
\label{Mendel_coefficient_determination_SS_tot}
\end{equation}
%
\item $\circ$ $SS_{err}$ is the residual sum of squares,
as given in equation(\ref{Mendel_coefficient_determination_SS_err})
\cite{Gujarati2004,Montgomery2003}, which is the objective of
the traditional OLS estimation, and the OLS method is to minimise
the $SS_{err}$.

%
\begin{equation}
SS_{err} = \sum\limits_{i = 1}^k{ \left( y_i - \hat{y}_i \right)^2}
\label{Mendel_coefficient_determination_SS_err}
\end{equation}
%
\end{list}


Meanwhile, $R^2$ also measures the
proportion of the variance in the dependent variable explained by
the independent variable \cite{Allen2007}, so that, equation (\ref{Mendel_coefficient_determination}) can be re-written as equation equation (\ref{Mendel_coefficient_determination_variance}) in terms of the variances, which compares the
explained variance $\sigma^2_{e}$
with the total variance $\sigma^2_{y}$.

\begin{equation}
R^2 \left( y_i,\hat{y}_i \right)  = 1 - \displaystyle
\frac{\sigma^2_{e}}{\sigma^2_{y}} 
%= 1 - \displaystyle \frac{
%\sum\limits_{i = 1}^k{ \left( y_i - \hat{y}_i
%\right)^2}}{\sum\limits_{i =1}^k{ \left( y_i - \bar{y}_i \right)^2}}
%R^2 \left( y_i,\hat{y}_i \right)  = 1 - \displaystyle
%\frac{\sigma^2_{e}}{\sigma^2_{y}} = \displaystyle \frac{
%\displaystyle \frac{\sum\limits_{i = 1}^k{ \left( y_i - \hat{y}_i
%\right)^2}}{k}}{\displaystyle \frac{\sum\limits_{i =1}^k{ \left( y_i
%- \bar{y}_i \right)^2}}{k}}
\label{Mendel_coefficient_determination_variance}
\end{equation}

where,
\begin{list}{$$}
\item $\circ$ $\sigma^2_{e}$ is the explained variance, which is the variance of the model's
predictions, as given in equation
(\ref{Mendel_coefficient_determination_sigma_e});
%
\begin{equation}
\sigma^2_{e} = \displaystyle \frac{\sum\limits_{i = 1}^k{ \left( y_i
- \hat{y}_i \right)^2}}{k} = \displaystyle \frac{SS_{err}}{k}
\label{Mendel_coefficient_determination_sigma_e}
\end{equation}
%
\item $\circ$ $\sigma^2_{y}$ is the total variance,
as given in equation(\ref{Mendel_coefficient_determination_sigma_y});
%
\begin{equation}
\sigma^2_{y} = \displaystyle \frac{ \sum\limits_{i = 1}^k{ \left(
y_i - \bar{y}_i \right)^2}}{k} = \displaystyle \frac{SS_{tot}}{k}
\label{Mendel_coefficient_determination_sigma_y}
\end{equation}
\end{list}

%http://en.wikipedia.org/wiki/Coefficient_of_determination
%http://www.mathworks.com/access/helpdesk/help/techdoc/index.html?/access/helpdesk/help/techdoc/learn_matlab/bq45say-1.html

\subsection{ Fitness Function of Regression Modelling }
\label{Fitness Function Design}

To order to evaluate the performance of the Mendel-GA comparison
process with the existing exchange rates and COF feature, the $R^2$
has been borrowed for the fitness function definition, which is a
statistical measurement of the agreement between the observed data
sequence $y_i$ and the Mendel-GA generated specimen data sequence
$\hat{y}_i$.


%http://en.wikipedia.org/wiki/Coefficient_of_determination
%http://www.mathworks.com/access/helpdesk/help/techdoc/index.html?/access/helpdesk/help/techdoc/learn_matlab/bq45say-1.html
%$\varepsilon$ is a small value parameter and $\varepsilon$ $\rightarrow$ 0, which helps to avoid $J$ $\rightarrow$ $\infty $.

%SGALAB Mendel-GA.vsd

\begin{figure}[h!]
\centering
\includegraphics[scale=0.17]{Mendel_orginal_data_prehandling}
\caption{ Original data pre-handling process and fitness function
generation} \label{Mendel_orginal_data_prehandling}
\end{figure}

Using the basic idea of the $R^2$, the association between exchange
rates and COF is specified as equation \ref{Mendel_regression_exp},
we define the fitness function as equation
\ref{Mendel_regression_fitness}, which aims to use the Mendel-GA to
find the best parameters that make the $R^2$ approaching to the
maximal.

\begin{equation}
\Delta \hat{y}_i(t) = \beta_1 + \beta_2 \Delta x_i(t) + \beta_3
\Delta y_i(t-1)\label{Mendel_regression_exp}
\end{equation}

\begin{equation}
J = fitness(\beta_1, \beta_2, \beta_3) \rightarrow R^2(\Delta
\hat{y}_i, \Delta {y}_i) \label{Mendel_regression_fitness}
\end{equation}

where,
\begin{list}{$$}
%
\item $\circ$ $\Delta \hat{y}_i(t)$ is the estimated(modelling) value from the
regression model at time $t$;
%
\item $\circ$ $\beta_1$, $\beta_2$ and $\beta_3$ are known
as the intercept coefficient, slope coefficient and historical data
coefficient, respectively;
%
\item $\circ$ $x_i$ is the observed COF of a foreign currency;
%
\item $\circ$ $y_i$ defines the ratio of a foreign currency vs. US. dollar in
$\log(*)$ space. Equations (\ref{Mendel_yi_USD}) and
(\ref{Mendel_yi_JPY}) are the data scaling operations for the cases of Deutsche Mark vs. US. Dollar
and Japanese Yen vs. US. Dollar, respectively;
%
\begin{equation}
y_i =  \displaystyle  \log\left(\displaystyle
 \frac{DM}{USD}\right) \times 10000 \label{Mendel_yi_USD}
\end{equation}

\begin{equation}
y_i =  \displaystyle  \log\left(\displaystyle
 \frac{JPY}{USD}\right) \times 10000 \label{Mendel_yi_JPY}
\end{equation}

%
\item $\circ$ $\Delta x_i(t)$ and $\Delta y_i(t-1)$ are the difference data sequence
obtained from the pre-data handling process at time $t$ and $t-1$,
as stated in equations (\ref{Mendel_delta_xi}) and
(\ref{Mendel_delta_yi});

\begin{equation}
\Delta x_i(t) =  x_i(t) -  x_i(t-1)
\label{Mendel_delta_xi}
\end{equation}

\begin{equation}
\Delta y_i(t-1) =  y_i(t-1) -  y_i(t-2) \label{Mendel_delta_yi}
\end{equation}
%
\end{list}


As can be seen from equation (\ref{Mendel_coefficient_determination}), the numerical simulation generates tiny errors by ($y_i - \hat{y}_i$) and ($y_i - \bar{y}_i$) which could cause the loss of precision. By the scaling factor the numerical errors can be scaled to an acceptable numerical range for the computing precision protection. The scaling factor can be selected on a case-by-case basis. As shown in Figure \ref{Mendel_orginal_data_prehandling}, in order
to reduce the possibility of numerical simulation error, the data
sequence $y_i$ has been scaled to a quantity by times a scaling factor (a constant) 10000 for the cases of DM vs. USD and JPY vs. USD. Then, generate the difference data sequence of $\Delta x_i$
and $\Delta y_i$, and the next step is to generate the fitness
function by the equation (\ref{Mendel_regression_fitness}).



\section{Empirical Results and Discussions }
\label{Simulation}

The original exchange rates data came from Evans and Lyons
\cite{Evans2002} in 2002, in which there are two groups of trading
data:
\begin{list}{$$}
%
\item $\circ$ Data-I: First trading data pair of Deutsche Mark (DM) vs. US. Dollar
(USD); % - $\log(DM/USD)$ - $x_i$ vs. COF - $y_i$,
%
\item $\circ$ Data-II: Second trading data pair of Japanese Yen (JPY) vs. US. Dollar (USD);
% - $\log(JPY/USD)$ - $x_i$ vs. COF - $y_i$.
\end{list}

Empirical results are obtained using a specially devised simulation
toolkit of Mendel-GA for $MATLAB$, known henceforth here as $SGALAB$
\cite{sgalab2009}. Unless stated otherwise all the results are
generated using the following parameters of the genetic algorithms
as listed in Table \ref{tableGA}, in which binary encoding/decoding,
tournament selection, single point crossover and mutation are
utilised by the Mendel-GA evolutionary process. The Mendel-GA's
results are also compared to the results of standard GA and OLS
methods.

%Table 1. GA parameters
\begin{table}[h!]
\caption{Empirical parameters for Mendel-GA } % title of Table
\centering % used for centering table
\begin{tabular}{lll}
\hline
               & max generations                               & 100         \\
%\hline
               & crossover probability                        & 0.8          \\
%\hline
               & mutation probability                         & 0.001        \\
%\hline
               & population                                   & 50          \\
%\hline
               & selection operator                           & tournament   \\
%\hline
               & crossover operator                           & single point \\
%\hline
               & mutation operator                            & single point \\
%\hline
               & encoding method                              & binary     \\
%\hline
$\beta_1$   & regression parameter   & $[ -10 , 10 ]$  \\
%\hline
$\beta_2$   & regression parameter   & $[ 0 , 5]$  \\
%\hline
$\beta_3$   & regression parameter   & $[ -1 , 1]$  \\
%\hline
$ $         & Mendel percentage      & 1 (full chromosome length)  \\
%\hline
$ $         & number of simulations & 1000  \\
%\hline
\hline
\end{tabular}
\label{tableGA} % is used to refer this table in the text
\end{table}

\begin{figure}[h!]
\centering
\includegraphics[scale=1.0]{Mendel_fitness_dm}
\caption{Data-I (DM vs. USD) evolutionary fitness data }
\label{Mendel_fitness_dm_dollar}
\end{figure}



\begin{figure}[h!]
\centering
\includegraphics[scale=1.0]{Mendel_fitness_yen}
\caption{Data-II (JPY vs. USD) evolutionary fitness data}
\label{Mendel_fitness_yen_dollar}
\end{figure}




%Table 1. GA parameters
\begin{table}[h!]
\caption{Mendel-GA results of mean for data-I (DM vs. USD) } % title of Table
\centering % used for centering table
\begin{tabular}{llll}
\hline
%\hline
            & Mendel-GA  & SGA       & OLS \\
\hline
E[$\beta_1$]   & -4.9297  & -4.9251 &  -4.9143 \\
%\hline
E[$\beta_2$]   & 0.2211   & 0.2203  & 0.2194 \\
%\hline
E[$\beta_3$]   & 0.1490   & 0.1478  & 0.1399 \\
%\hline
E[$R^2$]       & 0.6520  & 0.6415 & 0.6400 \\
%\hline
\hline
\end{tabular}
\label{tableGAresults_DM_mean} % is used to refer this table in the text
\end{table}

%Table 1. GA parameters
\begin{table}[h!]
\caption{ Mendel-GA results of variance for data-I (DM vs. USD) } % title of Table
\centering % used for centering table
\begin{tabular}{llll}
\hline
%\hline
                 & Mendel-GA    & SGA          & OLS \\
\hline
VAR[$\beta_1$]   & 0.2341       & 0.2572       & - \\
%\hline
VAR[$\beta_2$]   & 0.00233      & 0.00242      & - \\
%\hline
VAR[$\beta_3$]   & 0.0013       & 0.0018       & - \\
%\hline
VAR[$R^2$]       & 0.012        & 0.016        & - \\
%\hline
\hline
\end{tabular}
\label{tableGAresults_DM_var} % is used to refer this table in the text
\end{table}


%Table 1. GA parameters
\begin{table}[h!]
\caption{ Mendel-GA results of mean for data-II (JPY vs. USD)} % title of Table
\centering % used for centering table
\begin{tabular}{llll}
\hline
              & Mendel-GA   & SGA     & OLS \\
\hline
E[$\beta_1$]   &  -4.9625   & -4.9012  & -4.8334 \\
%\hline
E[$\beta_2$]   &  0.3185    & 0.3035   & 0.2952 \\
%\hline
E[$\beta_3$]   &  -0.1411   & -0.1402  &  -0.1396\\
%\hline
E[$R^2$]       &   0.4073   &  0.4064  &  0.40 \\
%\hline
\hline
\end{tabular}
\label{tableGAresults_JPY_mean} % is used to refer this table in the text
\end{table}


%Table 1. GA parameters
\begin{table}[h!]
\caption{ Mendel-GA results of variance for data-II (JPY vs. USD) } % title of Table
\centering % used for centering table
\begin{tabular}{llll}
\hline
%\hline
                 & Mendel-GA  & SGA        & OLS \\
\hline
VAR[$\beta_1$]   & 0.2343     & 0.2532     &  -\\
%\hline
VAR[$\beta_2$]   & 0.00112    & 0.00127    &  - \\
%\hline
VAR[$\beta_3$]   & 0.0025     & 0.0031     &  - \\
%\hline
VAR[$R^2$]       & 0.0025     &  0.0037    &  - \\
%\hline
\hline
\end{tabular}
\label{tableGAresults_JPY_var} % is used to refer this table in the text
\end{table}


As listed in Table \ref{tableGA}, the total empirical number is
1000. For the fitness results, the max, min and mean values of
fitness are calculated to show the Mendel-GA's single-run
performance ($fitmax_i$,$fitmin_i$ and $fitmean_i$), and for the
Mendel-GA's 1000 total experiments, as shown in Figures
\ref{Mendel_fitness_dm_dollar} and \ref{Mendel_fitness_yen_dollar},
the plots for the average data of all the $fitmax_i$,$fitmin_i$ and
$fitmean_i$ over the 1000 experiments are expressed.

Meanwhile, for the Mendel-GA's 1000 total experiments, the mean E(*)
and the variance VAR(*) of the max fitness values of all experiments
are evaluated as the Mendel-GA's overall performance indexes, as
stated in Tables \ref{tableGAresults_DM_mean},
\ref{tableGAresults_DM_var}, \ref{tableGAresults_JPY_mean} and
\ref{tableGAresults_JPY_var}.


Figure \ref{Mendel_fitness_dm_dollar} and Figure
\ref{Mendel_fitness_yen_dollar} describe the evaluation process of
the Data-I (DM vs. USD) fitness values , and the Data-II (JPY vs.
USD) fitness values, respectively. Over the full evolution time
defined by max generation, the fitness values grow up quickly to a
steady status with the initial parameters in Table \ref{tableGA}.
The solid lines in Figure \ref{Mendel_fitness_dm_dollar} and Figure
\ref{Mendel_fitness_yen_dollar} are the mean value of the average of
total fitness, the fitness data marked as `+' in the top are max
value of the average of total fitness, and the fitness data marked
as `+' in the bottom are the min value of the average of total
fitness.

\begin{figure}[h!]
\centering
\includegraphics[scale=1.0]{Mendel_deltax_deltay_dm_dollar}
\caption{ Sampling and modelling $\Delta x_i$ vs. $\Delta y_i$ for
Data-I (DM vs. USD)} \label{Mendel_deltax_deltay_dm_dollar}
\end{figure}

\begin{figure}[h!]
\centering
\includegraphics[scale=1.0]{Mendel_deltax_deltay_yen_dollar}
\caption{Sampling and modelling $\Delta x_i$ vs. $\Delta y_i$ for
Data-II (JPY vs. USD)} \label{Mendel_deltax_deltay_yen_dollar}
\end{figure}

In Tables \ref{tableGAresults_DM_mean}, \ref{tableGAresults_DM_var},
the mean and variance results of Data-I (DM vs. USD) are listed.
Specifically, in Table \ref{tableGAresults_DM_mean}, the
E[$\beta_1$], E[$\beta_2$] and E[$\beta_3$] of Mendel-GA, standard
GA and OLS are close to each other respectively. Among three of the
E[$R^2$] of Mendel-GA, standard GA and OLS, the Mendel-GA's E[$R^2$]
is the largest, the standard GA's E[$R^2$] is the second largest and
OLS's E[$R^2$] is the smallest, which indicate that the Mendel-GA
outperforms the two others.

Table \ref{tableGAresults_DM_var} shows that the VAR[$\beta_1$],
VAR[$\beta_2$] and VAR[$\beta_3$] of Mendel-GA, standard GA and OLS
are slightly different to each other, and the Mendel-GA's VAR[$R^2$]
is smaller than the standard GA's, which means the Mendel-GA's
results stay in a smaller scattered range than the standard GA's.
The OLS's VAR[*] values are zero because it is solved not by the
traditional numerical methods, whose results are the same every
experiment.


Similarity, Table \ref{tableGAresults_JPY_mean} and
\ref{tableGAresults_JPY_var} are the mean and variance results of
Data-II (JPY vs. USD), which provide the evidence to show the
Mendel-GA's better searching ability than the standard GA and OLS.



As shown in Figure \ref{Mendel_deltax_deltay_dm_dollar} and Figure
\ref{Mendel_deltax_deltay_yen_dollar}, the `o' data points are from
the sampling data, `*' are the data points from regression modelling
in equation \ref{Mendel_regression_exp}, they both show how the
regression model estimating the exchange rates trading behaviour.
The estimation for $\beta_1$, $\beta_2$ and $\beta_3$ are given in
Tables \ref{tableGAresults_DM_mean} and Table
\ref{tableGAresults_JPY_mean}.

\begin{figure}[h!]
\centering
\includegraphics[scale=1.0]{Mendel_beta3_deltadm}
\caption{$\beta_3$ vs. $\Delta$ $\log(DM/USD)$ when ($\beta_1$ =
$-4.9297$, $\beta_2$ = $0.2211$ )} \label{Mendel_beta3_deltadm}
\end{figure}

\begin{figure}[h!]
\centering
\includegraphics[scale=1.0]{Mendel_beta3_deltayen}
\caption{$\beta_3$ vs. $\Delta$ $\log(JPY/USD)$ when ($\beta_1$ = $
-4.9625$, $\beta_2$ = $0.3185$ )} \label{Mendel_beta3_deltayen}
\end{figure}

According to the simulation results from Mendel-GA method, $\beta_1$
and $\beta_2$ are in a relative steady status, and $\beta_3$ shows
how the history data of $\Delta y_i(t-1)$ can lead the estimating
data $\Delta \hat{y}(t)$. The Figure \ref{Mendel_beta3_deltadm} and
Figure \ref{Mendel_beta3_deltayen} show how the $\beta_3$ taking
effects on the $R^2$, which measures the performance of regression
modelling.

As shown in Tables \ref{tableGAresults_DM_mean} and
\ref{tableGAresults_JPY_mean}, comparing with the study of Evans and
Lyons \cite{Evans2002}, the coefficient of determination get
improved: for dm/dollar, the coefficient of determination gets
improved from 60$\%$ $\rightarrow$ 64$\%$; and for yen/dollar, the
coefficient of determination gets a little bit improved, from 40$\%$
$\rightarrow$ 40.73$\%$.

\section{Conclusions}
\label{Conclusions}

Interestingly, we find the variation involved in the association
between $\beta_3$ and the coefficient of determination. In the case
of dm/dollar, $\beta_3$ is positive to make the coefficient of
determination get minimum while in the case of yen/dollar, $\beta_3$
is negative to make the coefficient of determination get minimum, which reflects different feedback trading behaviours in the different cases of the foreign exchange market.

For Data-I (DM vs. USD), it indicates a positive trend tracing
behaviour, which means over the sample period positive exchange
rates return induces buyer dominant trading behaviour. However, in
the case of Data-II (JPY vs. USD), the results suggest that a
positive exchange rates return induce a reverse trading behaviour.

According to the simulation results for Data-I (DM vs. USD) and
Data-II (JPY vs. USD), it is showing significant positive
association between exchanger rates return and the corresponding
order flow change, which is consistent with the theoretical
hypothesis.

Compared with the results of Mendel-GA, standard GA and OLS, it
indicates that the Mendel-GA outperformed the standard GA and OLS
methods, which add to the research efforts to bridge the divide
between macro and micro approaches to exchange rate economics by
examining the linkages between exchange rates movement, cumulative
order flow and expectations of macroeconomic variables.

\section{Future Works}

For trading decisions, the technical analysis is sometimes utilised
to assist traders make buying and selling decisions, which is trying
to take good advantage of the available time-series data. This paper
provides an evolutionary alternative of exchange rates
determination, which differs from conventional econometrical
methods, and we are trying to apply this proposed technique on some more trading date set in practices.

Meanwhile, the integration of experts' skills into this model is
considered to be essential for trading decision making purpose. In
further studies, fuzzy logic system will be used for developing
decision models in which the experience of a trader can be
incorporated in a natural or artificial way.

This paper's results suggest some interesting issues for further
investigation on the real-time trading policy making, in which the
$R^2$ will be one of the trading decision criteria, for example,
Value-at-Risk (VaR) with volatility predictions.


%\clearpage

%%%%%%%%%%%%%%%%%%%%%%%%%%%%%%%%%%%%%%%%%%%%%%%%%%%%%%%%%%%%%
\section*{Acknowledgements}
{\it 
The authors would like to acknowledge the partial supports provided by 
the National Natural Science Foundation of China (NSFC) No.51105061 and No. 30872183. 

Also, the authors would like to acknowledge the anonymous reviewers
with their valuable comments for this paper. }
%%%%%%%%%%%%%%%%%%%%%%%%%%%%%%%%%%%%%%%%%%%%%%%%%%%%%%%%%%%%%


% BibTeX users please use one of
%\bibliographystyle{spbasic}      % basic style, author-year citations
%\bibliographystyle{spmpsci}      % mathematics and physical sciences
%\bibliographystyle{spphys}       % APS-like style for physics
%\bibliography{}   % name your BibTeX data base
\bibliographystyle{alpha}
% Non-BibTeX users please use
\begin{thebibliography}{00}


%1
\bibitem{Obstfeld2000} M. Obstfeld, K. Rogoff,
             The six major puzzles in international macroeconomics: is there a common cause?, NBER Working Papers 7777 (2000).

%2
\bibitem{Rogoff1996} K. Rogoff,
             The purchasing power parity puzzle,
             Journal of Economic Literature 34 (1996) 647-668.

%3
\bibitem{Lyons2001} R. Lyons,
             The Microstructure Approach to Exchange Rates,
             London, England, MIT Press, 2001.

%4
\bibitem{Killeen2001} W.P. Killeen, R.K. Lyons, M.J. Moore ,
             Fixed versus flexible: lessons from ems order flow,
             Journal of International Money and Finance 25 (2006) 551-579.
%5
\bibitem{Payne2003} R. Payne,
             Informed trade in spot foreign exchange markets: an empirical investigation,
             Journal of International Economics, 61 (2003) 307-329.

%6
\bibitem{Hann1996} T.H. Hann, E. Steurer, Much ado about nothing? exchange rate forecasting: neural networks vs. linear models using
monthly and weekly data, Neurocomputing 10 (1996) 323-339.


%7
\bibitem{Qi2003} M. Qi and Y. Wu,
Nonlinear prediction of exchange rates with monetary
fundamentals, Journal of Empirical Finance 10 (2003) 623-640.

%8
\bibitem{Yadav2007} R.N. Yadav, P.K. Kalra, J. John,
Time series prediction with single multiplicative neuron model,
Applied Soft Computing 7 (2007) 1157-1163.


%9
\bibitem{Rimcharoen2005}
S. Rimcharoen, D. Sutivong, P. Chongstitvatana, Prediction of the
stock exchange of thailand using adaptive evolution strategies,
Proceedings of the 17th IEEE International Conference on Tools with
Artificial Intelligence, IEEE Computer Society, Washington, DC, USA
(2005)

%10
\bibitem{Worasucheep2009} C. Worasucheep and P. Chongstitvatana,
A multi-strategy differential evolution algorithm for financial,
Prediction with Single Multiplicative Neuron, ICONIP 2009, Part
II, LNCS 5864 (2009) 122-130. 

%11
\bibitem{Nenortaite2004} J. Nenortaite, R. Simutis,
Stocks' Trading System Based on the Particle Swarm Optimization
Algorithm, Workshop on Computational Methods in Finance and
Insurance, Springer, Lecture Notes in Computer Science 3039/2004 (2004) 843-850.

%12
\bibitem{Zhao2009}
L. Zhao, Y. Yang, Expert systems with applications: pso-based
single multiplicative neuron model for time series prediction,
Expert Systems with Applications 36 (2009) 2805-2812.

%13
\bibitem{Holland1975} J.H. Holland,
         Adaptation in Natural and Artificial Systems,
              the University of Michigan Press, 1975.

% % % % % % % % % % % % % % % % % % % % % % % % % % % % % % %
\bibitem{Chen_JAP_2011} Yi Chen, Yong Ma, Zheng Lu, Bei Peng, Qin Chen,
         Quantitative Analysis of Terahertz Spectra for Illicit Drugs using Adaptive-range Micro-genetic Algorithm,
         Journal of Applied Physics, 
         Volume 110, Issue 4, Pages 044902-10, (2011)
              
\bibitem{Chen_AES_2011}
Yi Chen, Yong Ma, Zheng Lu, Lixia Qiu, Jin He, Terahertz Spectroscopic Uncertainty Analysis for Explosive Mixture Components Determination using Multi-objective Micro Genetic Algorithm, Advances in Engineering Software, Volume 42, Issue 9, Pages 649-659, (2011)

\bibitem{Chen_OE_2011}
Yi Chen, Yong Ma, Zheng Lu, Zhi-Ning Xia, Hong Cheng, Chemical Components Determination via Terahertz Spectroscopic Statistical Analysis using Micro Genetic Algorithm, Optical Engineering, Volume 50, Issue 3, Pages 034401-12, (2011)

\bibitem{Chen_CEC_2007}
Y. Chen, M.P. Cartmell, Multi-objective Optimisation on Motorized Momentum Exchange Tether for Payload Orbital Transfer, 2007 IEEE Congress on Evolutionary Computation (CEC), 25-28 September, Singapore, (2007) 
% % % % % % % % % % % % % % % % % % % % % % % % % % % % % % % % 
%14
%http://ieeexplore.ieee.org/xpls/abs_all.jsp?arnumber=1007055
\bibitem{Mendel1865} G. Mendel,
            Experiments in Plant Hybridization,
            url{www.mendelweb.org/Mendel.html} (1865).

%15
\bibitem{O'Neil2009} D. O'Neil,
         Mendel's Genetics ,
         \url{http://anthro.palomar.edu/mendel/mendel_1.htm} (2009).

%16
\bibitem{Song1999} I.S. Song, H.W. Woo, M.J. Tahk,
            A genetic algorithm with a Mendel operator for global minimization,
             Proceedings of the 1999 Congress on Evolutionary Computation,
             Washington, DC, USA,
             2 (1999) 1521-1526. 
%http://ieeexplore.ieee.org/xpls/abs_all.jsp?arnumber=782664

%17
\bibitem{Park2001} C.S. Park, H. Lee, H.C. Bang, M.J. Tahk,
            Modified mendel operation for multimodal function optimization,
             Proceedings of the 2001 Congress on Evolutionary Computation,
             Seoul, South Korea,
             2 (2001) 1388-1392. 
%http://ieeexplore.ieee.org/xpls/abs_all.jsp?arnumber=934353


%18
\bibitem{Kadrovach2001} B. Anthony Kadrovach, S.R. Michaud, J.B. Zydallis, G.B. Lamont, B. Secrest, D. Strong,
         Extending the simple genetic algorithm into multi-objective problems via mendelian pressure,
         2001 Genetic and Evolutionary Computation Conference,
         San Francisco,
         California,
          7-11 July,
1 (2001) 181-188.

%19
\bibitem{Kadrovach2002} B.A. Kadrovach, J.B. Zydallis, G.B. Lamont,
            Use of mendelian pressure in a multi-objective genetic algorithm,
             Proceedings of the 2002 Congress on Evolutionary Computation,
             Piscataway, New Jersey,
             12-17 May,
             1 (2002) 962-967.

\bibitem{Ishibuchi2002}
H. Ishibuchi, T. Yoshida, T. Murata, Balance between genetic search and local search in memetic algorithms for multiobjective permutation flowshop scheduling, IEEE Transactions on Evolutionary Computation 7 (2003) 204-223. 

\bibitem{Furi2005}
Victoria Furi$\acute{o}$, Andr$\acute{e}$s Moya, Rafael Sanju$\acute{a}$n, The cost of replication fidelity in an RNA virus, PNAS 102 (2005) 10233-10237. 


\bibitem{Zitzler1999}
E. Zitzler, Evolutionary Algorithms for Multiobjective Optimization: Methods and Applications, Diss. ETH No. 13398, ETH Zurich, Switzerland, 1999.

\bibitem{NAS2004}
the National Academy of Sciences (NAS),
Teaching about Evolution and the Nature of Science,National Academy of Sciences Press, 2004.

\bibitem{Haupt2004}
R.L. Haupt, S.E. Haupt, Practical Genetic Algorithms, Second Edition, John Wiley $\&$ Sons, Inc., 2004.

%20
\bibitem{MOGA1993} C.M. Fonseca, P.J. Fleming,
    Genetic algorithm for multi-objective optimisation: formulation,
             discussion and generalization,
             In: S. Forrest (ed.),
             Genetic Algorithm: Proceedings of the Fifth
             International Conference, Morgan Kaufmann, San Mateo, CA,
             1 (1993) 141-153.

%21
\bibitem{NPGA1994} J.N. Horn,  D.E.N. Goldberg,
A niched pareto genetic algorithm for multi-objective
optimisation,
             Proceedings of the First IEEE Conference on Evolutionary Computation, IEEE World Congress on Computational              Intelligence,  27-29 Jun,
              1 (1994) 82-87.

%22
\bibitem{NSGA1994} N. Srinivas, K. Deb,
 Multiobjective optimisation using nondominated sorting in genetic
             algorithm,
             Evolutionary Computation 2 (1994) 221-248.

%23
\bibitem{NSGAII2002} K. Deb, A. Pratap, S. Agarwal, T. Meyarivan,
A fast and elitist multiobjective genetic algorithm: nsgaii,
             IEEE Trans. on Evolutionary Comput. 6 (2002) 182-197.

%24
\bibitem{Tenti1996} P. Tenti,
Forecasting foreign exchange rates using recurrent neural
networks, Applied Artificial Intelligence 10 (1996) 567-581. 

%25
\bibitem{Gujarati2004} D.N. Gujarati, Basic Econometrics(4th Edition),
McGraw-Hill Companies, 2004.

%26
\bibitem{Reitz2007}
S. Reitz, M.A. Schmidt, M.P. Taylor, End-user order flow and exchange rate dynamics, The European Journal of Finance 17 (2011) 153-168.  

%27
\bibitem{Rime2008} D. Rime, L. Sarno, E. Sojli,
 Exchange rate forecasting, order flow, and macroeconomic information, CEPR Discussion Paper No. DP7225. Available at SSRN: http://ssrn.com/abstract=1372545, 2009.

%28
\bibitem{Andersen2003} T. Andersen, T. Bollerslev, F.X. Diebold,
C. Vega, Micro effects of macro announcements: real-time price
discovery in foreign exchange, American Economic Review 93 (2003) 38-62.

%29
\bibitem{Bacchetta2006} P. Bacchetta, E. Wincoop,
Can information heterogeneity explain the exchange rate determination puzzle?,
American Economic Review 96 (2006) 552-576. 

%30
\bibitem{Marks2008} J. Marks,
The construction of mendel's laws, Evolutionary Anthropology 17 (2008) 250-253.

%31
\bibitem{Montgomery2003} D.C. Montgomery, G.C. Runger, Applied Statistics and Probability for Engineers(Third Edition), John Wiley $\&$ Sons, Inc., 2003.

%32
\bibitem{Evans2002} M. Evans, R. Lyons,
Order flow and exchange rate dynamics, Journal of Political
Economy, 110 (2002) 170-180. 

%33
\bibitem{Allen2007} M.P. Allen,
Understanding Regression Analysis, Springer, 2007.

%34
\bibitem{sgalab2009} Y. Chen,
Simple Genetic Algorithm Laboratory Toolbox for MATLAB,
\url{http://www.mathworks.co.uk/matlabcentral/fileexchange/5882}, 2009.

\end{thebibliography}

%\clearpage
%\section*{Appendix}
%\input{original_data_table.tex}

\clearpage

\clearpage

\tableofcontents

\clearpage

\listoffigures

\listoftables

\end{document}

\endinput
%%
%% End of file `elsarticle-template-harv.tex'.
